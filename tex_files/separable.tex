\documentclass[12pt]{article}
\usepackage{amsfonts}
\usepackage{amsthm}
\usepackage{amsmath}
\usepackage[mathscr]{euscript}
%\usepackage{esint}
%\usepackage[utf8]{inputenc}
%\usepackage[portuguese]{babel}


\def\be{\begin{equation}}
\def\ee{\end{equation}}
\def\bee{\begin{equation*}}
\def\eee{\end{equation*}}
\def\bea{\begin{eqnarray*}}
\def\eea{\end{eqnarray*}}
\def\beaa{\begin{eqnarray}}
\def\eeaa{\end{eqnarray}}

\def\f{\frac}

\def\R{\mathbb{R}}
\def\K{\mathbb{K}}
\def\C{\mathbb{C}}
\def\I{\mathbb{I}}
\def\Z{\mathbb{Z}}
\def\Q{\mathbb{Q}}
\def\N{\mathbb{N}}

\def\cd{\cdot}

\def\v#1{{\boldsymbol{#1}}}
\def\ve#1{\hat{\boldsymbol#1}}

\def\l{\left}
\def\r{\right}
\def\la{\l\langle}
\def\ra{\r\rangle}
\def\div{\nabla\cdot}
\def\curl{\nabla\times}
\def\grad{\nabla}
\def\lap{\nabla^2}

\def\s{\quad}
\def\ss{\qquad}
\def\infi{\infty}
\def\p{\partial}
\def\u{\cup}%union of two sets
\def\i{\cap}%intersection of two sets
\def\ds{\oplus}
 
\newtheorem{theorem}{Theorem}[section]
\newtheorem{corollary}{Corollary}[theorem]
\newtheorem{lemma}[theorem]{Lemma}
\theoremstyle{definition}
\newtheorem{definition}{Definition}[section]
\newtheorem*{remark}{Remark}
\newtheorem{method}{Method}[section]
\newtheorem{example}{Example}[section]
\newtheorem{proposition}{Proposition}[section]


%\newenvironment{defi}
%  {\renewcommand{\qedsymbol}{$\whitesquare$}
%   \pushQED{\qed}\begin{defi/}}
%  {\popQED\end{defi/}}

\renewcommand\qedsymbol{$Q.E.D$}

\numberwithin{equation}{section}

\newcommand\norm[1]{\left\lVert#1\right\rVert}

\DeclareMathOperator{\sech}{sech}
\DeclareMathOperator{\csch}{csch}
\DeclareMathOperator{\asinh}{asinh}
\DeclareMathOperator{\atanh}{atanh}
\DeclareMathOperator{\acoth}{acoth}
\DeclareMathOperator{\acosh}{acosh}
\DeclareMathOperator{\acsch}{acsch}
\DeclareMathOperator{\asech}{asech}
\DeclareMathOperator{\im}{Im}
\DeclareMathOperator{\D}{D}
\DeclareMathOperator{\Tr}{tr}
\DeclareMathOperator{\graph}{graph}
\DeclareMathOperator{\spann}{span}
\DeclareMathOperator{\aut}{Aut}
%\DeclareMathOperator{\det}{det}
%\DeclareMathOperator{\dim}{dim}

\title{Green's Function Solution for Many Functions}
\author{Lucas Froguel\\Deparment of Physics, UFPR}

\begin{document}
\maketitle



        \section{Caso 1}
        Esse caso considerará as funções
        \bea
            f &=& 0 \\
            g &=& 0
        \eea
        Isso significa que precisamos resolver as EDOs:
        \be
            C X{\left(x \right)} + \frac{d^{2}}{d x^{2}} X{\left(x \right)} = 0
        \ee
        E
        \be
            - C Z{\left(z \right)} + \frac{d^{2}}{d z^{2}} Z{\left(z \right)} = 0
        \ee
        Isso resulta na solução parcial:
        \bea
            X &=& C_{1} e^{- x \sqrt{- C}} + C_{2} e^{x \sqrt{- C}} \\
            Z &=& C_{1} e^{- \sqrt{C} z} + C_{2} e^{\sqrt{C} z}
        \eea
        que implica no perfil:
        \be
            \Psi = \left(C_{1} e^{- \sqrt{C} z} + C_{2} e^{\sqrt{C} z}\right) \left(C_{1} e^{- x \sqrt{- C}} + C_{2} e^{x \sqrt{- C}}\right)
        \ee
        Isso leva aos campos magnéticos:
        \bea
            B_z &=& \left(C_{1} e^{- \sqrt{C} z} + C_{2} e^{\sqrt{C} z}\right) \left(- C_{1} \sqrt{- C} e^{- x \sqrt{- C}} + C_{2} \sqrt{- C} e^{x \sqrt{- C}}\right)\\
            B_x &=& - \left(C_{1} e^{- x \sqrt{- C}} + C_{2} e^{x \sqrt{- C}}\right) \left(- \sqrt{C} C_{1} e^{- \sqrt{C} z} + \sqrt{C} C_{2} e^{\sqrt{C} z}\right)
        \eea
        

        \section{Caso 2}
        Esse caso considerará as funções
        \bea
            f &=& 1 \\
            g &=& 1
        \eea
        Isso significa que precisamos resolver as EDOs:
        \be
            C X{\left(x \right)} + \mu_{0}^{2} + \frac{d^{2}}{d x^{2}} X{\left(x \right)} = 0
        \ee
        E
        \be
            - C Z{\left(z \right)} + \mu_{0} e^{- \frac{g z}{R T}} + \frac{d^{2}}{d z^{2}} Z{\left(z \right)} = 0
        \ee
        Isso resulta na solução parcial:
        \bea
            X &=& C_{1} e^{- x \sqrt{- C}} + C_{2} e^{x \sqrt{- C}} - \frac{\mu_{0}^{2}}{C} \\
            Z &=& C_{1} e^{- \sqrt{C} z} + C_{2} e^{\sqrt{C} z} - \frac{R^{2} T^{2} \mu_{0} e^{- \frac{g z}{R T}}}{- C R^{2} T^{2} + g^{2}}
        \eea
        que implica no perfil:
        \be
            \Psi = \left(C_{1} e^{- \sqrt{C} z} + C_{2} e^{\sqrt{C} z} - \frac{R^{2} T^{2} \mu_{0} e^{- \frac{g z}{R T}}}{- C R^{2} T^{2} + g^{2}}\right) \left(C_{1} e^{- x \sqrt{- C}} + C_{2} e^{x \sqrt{- C}} - \frac{\mu_{0}^{2}}{C}\right)
        \ee
        Isso leva aos campos magnéticos:
        \bea
            B_z &=& \left(- C_{1} \sqrt{- C} e^{- x \sqrt{- C}} + C_{2} \sqrt{- C} e^{x \sqrt{- C}}\right) \left(C_{1} e^{- \sqrt{C} z} + C_{2} e^{\sqrt{C} z} - \frac{R^{2} T^{2} \mu_{0} e^{- \frac{g z}{R T}}}{- C R^{2} T^{2} + g^{2}}\right)\\
            B_x &=& - \left(C_{1} e^{- x \sqrt{- C}} + C_{2} e^{x \sqrt{- C}} - \frac{\mu_{0}^{2}}{C}\right) \left(- \sqrt{C} C_{1} e^{- \sqrt{C} z} + \sqrt{C} C_{2} e^{\sqrt{C} z} + \frac{R T \mu_{0} g e^{- \frac{g z}{R T}}}{- C R^{2} T^{2} + g^{2}}\right)
        \eea
        

        \section{Caso 3}
        Esse caso considerará as funções
        \bea
            f &=& x \\
            g &=& 0
        \eea
        Isso significa que precisamos resolver as EDOs:
        \be
            C X{\left(x \right)} + \mu_{0}^{2} x + \frac{d^{2}}{d x^{2}} X{\left(x \right)} = 0
        \ee
        E
        \be
            - C Z{\left(z \right)} + \frac{d^{2}}{d z^{2}} Z{\left(z \right)} = 0
        \ee
        Isso resulta na solução parcial:
        \bea
            X &=& C_{1} e^{- x \sqrt{- C}} + C_{2} e^{x \sqrt{- C}} - \frac{\mu_{0}^{2} x}{C} \\
            Z &=& C_{1} e^{- \sqrt{C} z} + C_{2} e^{\sqrt{C} z}
        \eea
        que implica no perfil:
        \be
            \Psi = \left(C_{1} e^{- \sqrt{C} z} + C_{2} e^{\sqrt{C} z}\right) \left(C_{1} e^{- x \sqrt{- C}} + C_{2} e^{x \sqrt{- C}} - \frac{\mu_{0}^{2} x}{C}\right)
        \ee
        Isso leva aos campos magnéticos:
        \bea
            B_z &=& \left(C_{1} e^{- \sqrt{C} z} + C_{2} e^{\sqrt{C} z}\right) \left(- C_{1} \sqrt{- C} e^{- x \sqrt{- C}} + C_{2} \sqrt{- C} e^{x \sqrt{- C}} - \frac{\mu_{0}^{2}}{C}\right)\\
            B_x &=& - \left(- \sqrt{C} C_{1} e^{- \sqrt{C} z} + \sqrt{C} C_{2} e^{\sqrt{C} z}\right) \left(C_{1} e^{- x \sqrt{- C}} + C_{2} e^{x \sqrt{- C}} - \frac{\mu_{0}^{2} x}{C}\right)
        \eea
        

        \section{Caso 4}
        Esse caso considerará as funções
        \bea
            f &=& X{\left(x \right)} \\
            g &=& 0
        \eea
        Isso significa que precisamos resolver as EDOs:
        \be
            C X{\left(x \right)} + \mu_{0}^{2} X{\left(x \right)} + \frac{d^{2}}{d x^{2}} X{\left(x \right)} = 0
        \ee
        E
        \be
            - C Z{\left(z \right)} + \frac{d^{2}}{d z^{2}} Z{\left(z \right)} = 0
        \ee
        Isso resulta na solução parcial:
        \bea
            X &=& C_{1} e^{- x \sqrt{- C - \mu_{0}^{2}}} + C_{2} e^{x \sqrt{- C - \mu_{0}^{2}}} \\
            Z &=& C_{1} e^{- \sqrt{C} z} + C_{2} e^{\sqrt{C} z}
        \eea
        que implica no perfil:
        \be
            \Psi = \left(C_{1} e^{- \sqrt{C} z} + C_{2} e^{\sqrt{C} z}\right) \left(C_{1} e^{- x \sqrt{- C - \mu_{0}^{2}}} + C_{2} e^{x \sqrt{- C - \mu_{0}^{2}}}\right)
        \ee
        Isso leva aos campos magnéticos:
        \bea
            B_z &=& \left(C_{1} e^{- \sqrt{C} z} + C_{2} e^{\sqrt{C} z}\right) \left(- C_{1} \sqrt{- C - \mu_{0}^{2}} e^{- x \sqrt{- C - \mu_{0}^{2}}} + C_{2} \sqrt{- C - \mu_{0}^{2}} e^{x \sqrt{- C - \mu_{0}^{2}}}\right)\\
            B_x &=& - \left(C_{1} e^{- x \sqrt{- C - \mu_{0}^{2}}} + C_{2} e^{x \sqrt{- C - \mu_{0}^{2}}}\right) \left(- \sqrt{C} C_{1} e^{- \sqrt{C} z} + \sqrt{C} C_{2} e^{\sqrt{C} z}\right)
        \eea
        

        \section{Caso 5}
        Esse caso considerará as funções
        \bea
            f &=& x^{3} + x^{2} + x + 1 \\
            g &=& 0
        \eea
        Isso significa que precisamos resolver as EDOs:
        \be
            C X{\left(x \right)} + \mu_{0}^{2} \left(x^{3} + x^{2} + x + 1\right) + \frac{d^{2}}{d x^{2}} X{\left(x \right)} = 0
        \ee
        E
        \be
            - C Z{\left(z \right)} + \frac{d^{2}}{d z^{2}} Z{\left(z \right)} = 0
        \ee
        Isso resulta na solução parcial:
        \bea
            X &=& C_{1} e^{- x \sqrt{- C}} + C_{2} e^{x \sqrt{- C}} - \frac{\mu_{0}^{2} x^{3}}{C} - \frac{\mu_{0}^{2} x^{2}}{C} - \frac{\mu_{0}^{2} x}{C} - \frac{\mu_{0}^{2}}{C} + \frac{6 \mu_{0}^{2} x}{C^{2}} + \frac{2 \mu_{0}^{2}}{C^{2}} \\
            Z &=& C_{1} e^{- \sqrt{C} z} + C_{2} e^{\sqrt{C} z}
        \eea
        que implica no perfil:
        \be
            \Psi = \left(C_{1} e^{- \sqrt{C} z} + C_{2} e^{\sqrt{C} z}\right) \left(C_{1} e^{- x \sqrt{- C}} + C_{2} e^{x \sqrt{- C}} - \frac{\mu_{0}^{2} x^{3}}{C} - \frac{\mu_{0}^{2} x^{2}}{C} - \frac{\mu_{0}^{2} x}{C} - \frac{\mu_{0}^{2}}{C} + \frac{6 \mu_{0}^{2} x}{C^{2}} + \frac{2 \mu_{0}^{2}}{C^{2}}\right)
        \ee
        Isso leva aos campos magnéticos:
        \bea
            B_z &=& \left(C_{1} e^{- \sqrt{C} z} + C_{2} e^{\sqrt{C} z}\right) \left(- C_{1} \sqrt{- C} e^{- x \sqrt{- C}} + C_{2} \sqrt{- C} e^{x \sqrt{- C}} - \frac{3 \mu_{0}^{2} x^{2}}{C} - \frac{2 \mu_{0}^{2} x}{C} - \frac{\mu_{0}^{2}}{C} + \frac{6 \mu_{0}^{2}}{C^{2}}\right)\\
            B_x &=& - \left(- \sqrt{C} C_{1} e^{- \sqrt{C} z} + \sqrt{C} C_{2} e^{\sqrt{C} z}\right) \left(C_{1} e^{- x \sqrt{- C}} + C_{2} e^{x \sqrt{- C}} - \frac{\mu_{0}^{2} x^{3}}{C} - \frac{\mu_{0}^{2} x^{2}}{C} - \frac{\mu_{0}^{2} x}{C} - \frac{\mu_{0}^{2}}{C} + \frac{6 \mu_{0}^{2} x}{C^{2}} + \frac{2 \mu_{0}^{2}}{C^{2}}\right)
        \eea
        

        \section{Caso 6}
        Esse caso considerará as funções
        \bea
            f &=& x^{4} \\
            g &=& 0
        \eea
        Isso significa que precisamos resolver as EDOs:
        \be
            C X{\left(x \right)} + \mu_{0}^{2} x^{4} + \frac{d^{2}}{d x^{2}} X{\left(x \right)} = 0
        \ee
        E
        \be
            - C Z{\left(z \right)} + \frac{d^{2}}{d z^{2}} Z{\left(z \right)} = 0
        \ee
        Isso resulta na solução parcial:
        \bea
            X &=& C_{1} e^{- x \sqrt{- C}} + C_{2} e^{x \sqrt{- C}} - \frac{\mu_{0}^{2} x^{4}}{C} + \frac{12 \mu_{0}^{2} x^{2}}{C^{2}} - \frac{24 \mu_{0}^{2}}{C^{3}} \\
            Z &=& C_{1} e^{- \sqrt{C} z} + C_{2} e^{\sqrt{C} z}
        \eea
        que implica no perfil:
        \be
            \Psi = \left(C_{1} e^{- \sqrt{C} z} + C_{2} e^{\sqrt{C} z}\right) \left(C_{1} e^{- x \sqrt{- C}} + C_{2} e^{x \sqrt{- C}} - \frac{\mu_{0}^{2} x^{4}}{C} + \frac{12 \mu_{0}^{2} x^{2}}{C^{2}} - \frac{24 \mu_{0}^{2}}{C^{3}}\right)
        \ee
        Isso leva aos campos magnéticos:
        \bea
            B_z &=& \left(C_{1} e^{- \sqrt{C} z} + C_{2} e^{\sqrt{C} z}\right) \left(- C_{1} \sqrt{- C} e^{- x \sqrt{- C}} + C_{2} \sqrt{- C} e^{x \sqrt{- C}} - \frac{4 \mu_{0}^{2} x^{3}}{C} + \frac{24 \mu_{0}^{2} x}{C^{2}}\right)\\
            B_x &=& - \left(- \sqrt{C} C_{1} e^{- \sqrt{C} z} + \sqrt{C} C_{2} e^{\sqrt{C} z}\right) \left(C_{1} e^{- x \sqrt{- C}} + C_{2} e^{x \sqrt{- C}} - \frac{\mu_{0}^{2} x^{4}}{C} + \frac{12 \mu_{0}^{2} x^{2}}{C^{2}} - \frac{24 \mu_{0}^{2}}{C^{3}}\right)
        \eea
        

        \section{Caso 7}
        Esse caso considerará as funções
        \bea
            f &=& X{\left(x \right)} \\
            g &=& Z{\left(z \right)}
        \eea
        Isso significa que precisamos resolver as EDOs:
        \be
            C X{\left(x \right)} + \mu_{0}^{2} X{\left(x \right)} + \frac{d^{2}}{d x^{2}} X{\left(x \right)} = 0
        \ee
        E
        \be
            - C Z{\left(z \right)} + \mu_{0} Z{\left(z \right)} e^{- \frac{g z}{R T}} + \frac{d^{2}}{d z^{2}} Z{\left(z \right)} = 0
        \ee
        Isso resulta na solução parcial:
        \bea
            X &=& C_{1} e^{- x \sqrt{- C - \mu_{0}^{2}}} + C_{2} e^{x \sqrt{- C - \mu_{0}^{2}}} \\
            Z &=& C_{2} \left(\frac{C z^{2}}{2} - \frac{\mu_{0} z^{2} e^{- \frac{g z}{R T}}}{2} + \frac{z^{4} \left(C e^{\frac{g z}{R T}} - \mu_{0}\right)^{2} e^{- \frac{2 g z}{R T}}}{24} + 1\right) + C_{1} z \left(\frac{C z^{2}}{6} - \frac{\mu_{0} z^{2} e^{- \frac{g z}{R T}}}{6} + 1\right) + O\left(z^{6}\right)
        \eea
        que implica no perfil:
        \be
            \Psi = \left(C_{1} e^{- x \sqrt{- C - \mu_{0}^{2}}} + C_{2} e^{x \sqrt{- C - \mu_{0}^{2}}}\right) \left(C_{2} \left(\frac{C z^{2}}{2} - \frac{\mu_{0} z^{2} e^{- \frac{g z}{R T}}}{2} + \frac{z^{4} \left(C e^{\frac{g z}{R T}} - \mu_{0}\right)^{2} e^{- \frac{2 g z}{R T}}}{24} + 1\right) + C_{1} z \left(\frac{C z^{2}}{6} - \frac{\mu_{0} z^{2} e^{- \frac{g z}{R T}}}{6} + 1\right) + O\left(z^{6}\right)\right)
        \ee
        Isso leva aos campos magnéticos:
        \bea
            B_z &=& \left(- C_{1} \sqrt{- C - \mu_{0}^{2}} e^{- x \sqrt{- C - \mu_{0}^{2}}} + C_{2} \sqrt{- C - \mu_{0}^{2}} e^{x \sqrt{- C - \mu_{0}^{2}}}\right) \left(C_{2} \left(\frac{C z^{2}}{2} - \frac{\mu_{0} z^{2} e^{- \frac{g z}{R T}}}{2} + \frac{z^{4} \left(C e^{\frac{g z}{R T}} - \mu_{0}\right)^{2} e^{- \frac{2 g z}{R T}}}{24} + 1\right) + C_{1} z \left(\frac{C z^{2}}{6} - \frac{\mu_{0} z^{2} e^{- \frac{g z}{R T}}}{6} + 1\right) + O\left(z^{6}\right)\right)\\
            B_x &=& - \left(C_{1} e^{- x \sqrt{- C - \mu_{0}^{2}}} + C_{2} e^{x \sqrt{- C - \mu_{0}^{2}}}\right) \left(C_{2} \left(C z + \frac{C g z^{4} \left(C e^{\frac{g z}{R T}} - \mu_{0}\right) e^{- \frac{g z}{R T}}}{12 R T} - \mu_{0} z e^{- \frac{g z}{R T}} + \frac{z^{3} \left(C e^{\frac{g z}{R T}} - \mu_{0}\right)^{2} e^{- \frac{2 g z}{R T}}}{6} + \frac{\mu_{0} g z^{2} e^{- \frac{g z}{R T}}}{2 R T} - \frac{g z^{4} \left(C e^{\frac{g z}{R T}} - \mu_{0}\right)^{2} e^{- \frac{2 g z}{R T}}}{12 R T}\right) + C_{1} \left(\frac{C z^{2}}{6} - \frac{\mu_{0} z^{2} e^{- \frac{g z}{R T}}}{6} + 1\right) + C_{1} z \left(\frac{C z}{3} - \frac{\mu_{0} z e^{- \frac{g z}{R T}}}{3} + \frac{\mu_{0} g z^{2} e^{- \frac{g z}{R T}}}{6 R T}\right) + O\left(z^{5}\right)\right)
        \eea
        

        \section{Caso 8}
        Esse caso considerará as funções
        \bea
            f &=& \sin{\left(x \right)} + \cos{\left(x \right)} \\
            g &=& 0
        \eea
        Isso significa que precisamos resolver as EDOs:
        \be
            C X{\left(x \right)} + \mu_{0}^{2} \left(\sin{\left(x \right)} + \cos{\left(x \right)}\right) + \frac{d^{2}}{d x^{2}} X{\left(x \right)} = 0
        \ee
        E
        \be
            - C Z{\left(z \right)} + \frac{d^{2}}{d z^{2}} Z{\left(z \right)} = 0
        \ee
        Isso resulta na solução parcial:
        \bea
            X &=& C_{1} e^{- x \sqrt{- C}} + C_{2} e^{x \sqrt{- C}} - \frac{\mu_{0}^{2} \sin{\left(x \right)}}{C - 1} - \frac{\mu_{0}^{2} \cos{\left(x \right)}}{C - 1} \\
            Z &=& C_{1} e^{- \sqrt{C} z} + C_{2} e^{\sqrt{C} z}
        \eea
        que implica no perfil:
        \be
            \Psi = \left(C_{1} e^{- \sqrt{C} z} + C_{2} e^{\sqrt{C} z}\right) \left(C_{1} e^{- x \sqrt{- C}} + C_{2} e^{x \sqrt{- C}} - \frac{\mu_{0}^{2} \sin{\left(x \right)}}{C - 1} - \frac{\mu_{0}^{2} \cos{\left(x \right)}}{C - 1}\right)
        \ee
        Isso leva aos campos magnéticos:
        \bea
            B_z &=& \left(C_{1} e^{- \sqrt{C} z} + C_{2} e^{\sqrt{C} z}\right) \left(- C_{1} \sqrt{- C} e^{- x \sqrt{- C}} + C_{2} \sqrt{- C} e^{x \sqrt{- C}} + \frac{\mu_{0}^{2} \sin{\left(x \right)}}{C - 1} - \frac{\mu_{0}^{2} \cos{\left(x \right)}}{C - 1}\right)\\
            B_x &=& - \left(- \sqrt{C} C_{1} e^{- \sqrt{C} z} + \sqrt{C} C_{2} e^{\sqrt{C} z}\right) \left(C_{1} e^{- x \sqrt{- C}} + C_{2} e^{x \sqrt{- C}} - \frac{\mu_{0}^{2} \sin{\left(x \right)}}{C - 1} - \frac{\mu_{0}^{2} \cos{\left(x \right)}}{C - 1}\right)
        \eea
        \end{document}